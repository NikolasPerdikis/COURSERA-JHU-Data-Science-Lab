\documentclass[]{article}
\usepackage{lmodern}
\usepackage{amssymb,amsmath}
\usepackage{ifxetex,ifluatex}
\usepackage{fixltx2e} % provides \textsubscript
\ifnum 0\ifxetex 1\fi\ifluatex 1\fi=0 % if pdftex
  \usepackage[T1]{fontenc}
  \usepackage[utf8]{inputenc}
\else % if luatex or xelatex
  \ifxetex
    \usepackage{mathspec}
  \else
    \usepackage{fontspec}
  \fi
  \defaultfontfeatures{Ligatures=TeX,Scale=MatchLowercase}
\fi
% use upquote if available, for straight quotes in verbatim environments
\IfFileExists{upquote.sty}{\usepackage{upquote}}{}
% use microtype if available
\IfFileExists{microtype.sty}{%
\usepackage{microtype}
\UseMicrotypeSet[protrusion]{basicmath} % disable protrusion for tt fonts
}{}
\usepackage[margin=1in]{geometry}
\usepackage{hyperref}
\hypersetup{unicode=true,
            pdftitle={Statistical Inference Course Project \#1},
            pdfauthor={Nikolaos Perdikis},
            pdfborder={0 0 0},
            breaklinks=true}
\urlstyle{same}  % don't use monospace font for urls
\usepackage{color}
\usepackage{fancyvrb}
\newcommand{\VerbBar}{|}
\newcommand{\VERB}{\Verb[commandchars=\\\{\}]}
\DefineVerbatimEnvironment{Highlighting}{Verbatim}{commandchars=\\\{\}}
% Add ',fontsize=\small' for more characters per line
\usepackage{framed}
\definecolor{shadecolor}{RGB}{248,248,248}
\newenvironment{Shaded}{\begin{snugshade}}{\end{snugshade}}
\newcommand{\AlertTok}[1]{\textcolor[rgb]{0.94,0.16,0.16}{#1}}
\newcommand{\AnnotationTok}[1]{\textcolor[rgb]{0.56,0.35,0.01}{\textbf{\textit{#1}}}}
\newcommand{\AttributeTok}[1]{\textcolor[rgb]{0.77,0.63,0.00}{#1}}
\newcommand{\BaseNTok}[1]{\textcolor[rgb]{0.00,0.00,0.81}{#1}}
\newcommand{\BuiltInTok}[1]{#1}
\newcommand{\CharTok}[1]{\textcolor[rgb]{0.31,0.60,0.02}{#1}}
\newcommand{\CommentTok}[1]{\textcolor[rgb]{0.56,0.35,0.01}{\textit{#1}}}
\newcommand{\CommentVarTok}[1]{\textcolor[rgb]{0.56,0.35,0.01}{\textbf{\textit{#1}}}}
\newcommand{\ConstantTok}[1]{\textcolor[rgb]{0.00,0.00,0.00}{#1}}
\newcommand{\ControlFlowTok}[1]{\textcolor[rgb]{0.13,0.29,0.53}{\textbf{#1}}}
\newcommand{\DataTypeTok}[1]{\textcolor[rgb]{0.13,0.29,0.53}{#1}}
\newcommand{\DecValTok}[1]{\textcolor[rgb]{0.00,0.00,0.81}{#1}}
\newcommand{\DocumentationTok}[1]{\textcolor[rgb]{0.56,0.35,0.01}{\textbf{\textit{#1}}}}
\newcommand{\ErrorTok}[1]{\textcolor[rgb]{0.64,0.00,0.00}{\textbf{#1}}}
\newcommand{\ExtensionTok}[1]{#1}
\newcommand{\FloatTok}[1]{\textcolor[rgb]{0.00,0.00,0.81}{#1}}
\newcommand{\FunctionTok}[1]{\textcolor[rgb]{0.00,0.00,0.00}{#1}}
\newcommand{\ImportTok}[1]{#1}
\newcommand{\InformationTok}[1]{\textcolor[rgb]{0.56,0.35,0.01}{\textbf{\textit{#1}}}}
\newcommand{\KeywordTok}[1]{\textcolor[rgb]{0.13,0.29,0.53}{\textbf{#1}}}
\newcommand{\NormalTok}[1]{#1}
\newcommand{\OperatorTok}[1]{\textcolor[rgb]{0.81,0.36,0.00}{\textbf{#1}}}
\newcommand{\OtherTok}[1]{\textcolor[rgb]{0.56,0.35,0.01}{#1}}
\newcommand{\PreprocessorTok}[1]{\textcolor[rgb]{0.56,0.35,0.01}{\textit{#1}}}
\newcommand{\RegionMarkerTok}[1]{#1}
\newcommand{\SpecialCharTok}[1]{\textcolor[rgb]{0.00,0.00,0.00}{#1}}
\newcommand{\SpecialStringTok}[1]{\textcolor[rgb]{0.31,0.60,0.02}{#1}}
\newcommand{\StringTok}[1]{\textcolor[rgb]{0.31,0.60,0.02}{#1}}
\newcommand{\VariableTok}[1]{\textcolor[rgb]{0.00,0.00,0.00}{#1}}
\newcommand{\VerbatimStringTok}[1]{\textcolor[rgb]{0.31,0.60,0.02}{#1}}
\newcommand{\WarningTok}[1]{\textcolor[rgb]{0.56,0.35,0.01}{\textbf{\textit{#1}}}}
\usepackage{graphicx,grffile}
\makeatletter
\def\maxwidth{\ifdim\Gin@nat@width>\linewidth\linewidth\else\Gin@nat@width\fi}
\def\maxheight{\ifdim\Gin@nat@height>\textheight\textheight\else\Gin@nat@height\fi}
\makeatother
% Scale images if necessary, so that they will not overflow the page
% margins by default, and it is still possible to overwrite the defaults
% using explicit options in \includegraphics[width, height, ...]{}
\setkeys{Gin}{width=\maxwidth,height=\maxheight,keepaspectratio}
\IfFileExists{parskip.sty}{%
\usepackage{parskip}
}{% else
\setlength{\parindent}{0pt}
\setlength{\parskip}{6pt plus 2pt minus 1pt}
}
\setlength{\emergencystretch}{3em}  % prevent overfull lines
\providecommand{\tightlist}{%
  \setlength{\itemsep}{0pt}\setlength{\parskip}{0pt}}
\setcounter{secnumdepth}{0}
% Redefines (sub)paragraphs to behave more like sections
\ifx\paragraph\undefined\else
\let\oldparagraph\paragraph
\renewcommand{\paragraph}[1]{\oldparagraph{#1}\mbox{}}
\fi
\ifx\subparagraph\undefined\else
\let\oldsubparagraph\subparagraph
\renewcommand{\subparagraph}[1]{\oldsubparagraph{#1}\mbox{}}
\fi

%%% Use protect on footnotes to avoid problems with footnotes in titles
\let\rmarkdownfootnote\footnote%
\def\footnote{\protect\rmarkdownfootnote}

%%% Change title format to be more compact
\usepackage{titling}

% Create subtitle command for use in maketitle
\providecommand{\subtitle}[1]{
  \posttitle{
    \begin{center}\large#1\end{center}
    }
}

\setlength{\droptitle}{-2em}

  \title{Statistical Inference Course Project \#1}
    \pretitle{\vspace{\droptitle}\centering\huge}
  \posttitle{\par}
    \author{Nikolaos Perdikis}
    \preauthor{\centering\large\emph}
  \postauthor{\par}
      \predate{\centering\large\emph}
  \postdate{\par}
    \date{18 7 2019}


\begin{document}
\maketitle

\hypertarget{simulation-exercise}{%
\subsection{Simulation Exercise}\label{simulation-exercise}}

Exercise scope, rephrased from requirements: This project will
investigate the exponential distribution in R and compare it with the
Central Limit Theorem. The exponential distribution can be simulated in
R with rexp(n, lambda) where lambda is the rate parameter. The mean of
exponential distribution is 1/lambda and the standard deviation is also
1/lambda. Parameter lambda will be set to 0.2 for all of the
simulations. The distribution of averages of 40 exponentials is
investigated and for this, a thousand simulations will be needed.

First, create the distribution:

\begin{Shaded}
\begin{Highlighting}[]
\CommentTok{# Known parameters}
\NormalTok{lambda <-}\StringTok{ }\FloatTok{0.2}
\NormalTok{sims <-}\StringTok{ }\DecValTok{1000}
\CommentTok{#create uniqueness of our random generator}
\KeywordTok{set.seed}\NormalTok{(}\DecValTok{555}\NormalTok{);}
\CommentTok{# Generate 40 random exponentials}
\NormalTok{n <-}\StringTok{ }\KeywordTok{seq}\NormalTok{(}\DecValTok{0}\NormalTok{,}\DecValTok{40}\NormalTok{,}\DataTypeTok{length.out =} \DecValTok{1000}\NormalTok{)}
\CommentTok{# Define the Exponential Distro}
\NormalTok{my_dexp <-}\StringTok{ }\KeywordTok{dexp}\NormalTok{(n,}\DataTypeTok{rate=}\NormalTok{lambda)}
\CommentTok{#...and put it in a dataframe (required for ggplot)}
\NormalTok{df_dexp <-}\StringTok{ }\KeywordTok{data.frame}\NormalTok{(}\DataTypeTok{x=}\NormalTok{n,my_dexp)}
\end{Highlighting}
\end{Shaded}

\hypertarget{show-the-sample-mean-and-compare-it-to-the-theoretical-mean-of-the-distribution.}{%
\subsection{Show the sample mean and compare it to the theoretical mean
of the
distribution.}\label{show-the-sample-mean-and-compare-it-to-the-theoretical-mean-of-the-distribution.}}

The formulae for the computations below are available in Wikipedia
\url{https://en.wikipedia.org/wiki/Exponential_distribution}

\begin{Shaded}
\begin{Highlighting}[]
\CommentTok{# Based on link  }
\NormalTok{mu <-}\StringTok{ }\NormalTok{sigma <-}\StringTok{ }\DecValTok{1} \OperatorTok{/}\StringTok{ }\NormalTok{lambda}
\CommentTok{#Variance is the square of stdev}
\NormalTok{varExp <-}\StringTok{ }\NormalTok{sigma}\OperatorTok{^}\DecValTok{2}
\NormalTok{varExp}
\end{Highlighting}
\end{Shaded}

\begin{verbatim}
## [1] 25
\end{verbatim}

Let's plot the distribution:

\begin{Shaded}
\begin{Highlighting}[]
\KeywordTok{with}\NormalTok{(df_dexp, }\KeywordTok{plot}\NormalTok{(n, my_dexp,}\DataTypeTok{xlab=}\StringTok{"x"}\NormalTok{,}\DataTypeTok{ylab=}\StringTok{"Probability Density"}\NormalTok{))}
\KeywordTok{par}\NormalTok{(}\DataTypeTok{mar =} \KeywordTok{c}\NormalTok{(}\DecValTok{3}\NormalTok{, }\DecValTok{5}\NormalTok{, }\DecValTok{5}\NormalTok{, }\DecValTok{0}\NormalTok{))}
\KeywordTok{title}\NormalTok{(}\DataTypeTok{main =} \StringTok{"Exponential Distribution, lambda = 0.2"}\NormalTok{)}
\KeywordTok{text}\NormalTok{(}\DecValTok{0}\NormalTok{, lambda, }\DataTypeTok{labels =} \StringTok{"lambda"}\NormalTok{, }\DataTypeTok{pos =} \DecValTok{4}\NormalTok{)}
\KeywordTok{abline}\NormalTok{(}\DataTypeTok{v =}\NormalTok{ mu, }\DataTypeTok{col =} \StringTok{"red"}\NormalTok{)}
\NormalTok{muLabel <-}\StringTok{ }\KeywordTok{paste}\NormalTok{(}\StringTok{"mu ="}\NormalTok{, mu)}
\KeywordTok{text}\NormalTok{(mu, }\DecValTok{0}\NormalTok{, }\DataTypeTok{labels =}\NormalTok{ muLabel, }\DataTypeTok{pos =} \DecValTok{4}\NormalTok{, }\DataTypeTok{col =} \StringTok{"red"}\NormalTok{)}
\KeywordTok{abline}\NormalTok{(}\DataTypeTok{v =}\NormalTok{ varExp, }\DataTypeTok{col =} \StringTok{"blue"}\NormalTok{)}
\NormalTok{varExpLabel <-}\StringTok{ }\KeywordTok{paste}\NormalTok{(}\StringTok{"Variance ="}\NormalTok{, varExp)}
\KeywordTok{text}\NormalTok{(varExp, }\FloatTok{0.025}\NormalTok{, }\DataTypeTok{labels =}\NormalTok{ varExpLabel, }\DataTypeTok{pos =} \DecValTok{4}\NormalTok{, }
     \DataTypeTok{col =} \StringTok{"blue"}\NormalTok{)}
\end{Highlighting}
\end{Shaded}

\includegraphics{StatInfCP1_files/figure-latex/plotExponentialDistrib-1.pdf}

Result: So far we have calculated the mean mu of the distribution to be
5 and the variance to be 25

\hypertarget{show-how-variable-the-sample-is-via-variance-and-compare-it-to-the-theoretical-variance-of-the-distribution.}{%
\subsection{Show how variable the sample is (via variance) and compare
it to the theoretical variance of the
distribution.}\label{show-how-variable-the-sample-is-via-variance-and-compare-it-to-the-theoretical-variance-of-the-distribution.}}

On to create the 1000 simulated trials of the means of 40 exponentials:

\begin{Shaded}
\begin{Highlighting}[]
\KeywordTok{set.seed}\NormalTok{(}\DecValTok{575}\NormalTok{);}
\NormalTok{sm1Kn40 <-}\StringTok{ }\KeywordTok{replicate}\NormalTok{(}\DecValTok{1000}\NormalTok{,}\KeywordTok{mean}\NormalTok{(}\KeywordTok{rexp}\NormalTok{(}\DecValTok{40}\NormalTok{,lambda)))}
\KeywordTok{summary}\NormalTok{(sm1Kn40)}
\end{Highlighting}
\end{Shaded}

\begin{verbatim}
##    Min. 1st Qu.  Median    Mean 3rd Qu.    Max. 
##   3.124   4.505   5.019   5.039   5.538   7.884
\end{verbatim}

\ldots and get its measurements, so the mean, the variance using R
function var:

\begin{Shaded}
\begin{Highlighting}[]
\NormalTok{mu_sim <-}\StringTok{ }\KeywordTok{round}\NormalTok{(}\KeywordTok{mean}\NormalTok{(sm1Kn40),}\DecValTok{4}\NormalTok{)}
\NormalTok{VarSim <-}\StringTok{ }\KeywordTok{var}\NormalTok{(sm1Kn40)}
\end{Highlighting}
\end{Shaded}

\hypertarget{show-that-the-distribution-is-approximately-normal}{%
\subsection{Show that the distribution is approximately
normal:}\label{show-that-the-distribution-is-approximately-normal}}

Let's plot the histogram of the averages of the 40 exponentials,
simulated 1000 times:

\begin{Shaded}
\begin{Highlighting}[]
\KeywordTok{hist}\NormalTok{(sm1Kn40,}\DataTypeTok{xlab=}\StringTok{"Averages of Exponentials"}\NormalTok{,}\DataTypeTok{main=}\StringTok{"Distribution of Averages of Exponentials"}\NormalTok{)}
\KeywordTok{abline}\NormalTok{(}\DataTypeTok{v =} \KeywordTok{round}\NormalTok{(mu_sim,}\DecValTok{4}\NormalTok{), }\DataTypeTok{col =} \StringTok{"red"}\NormalTok{)}
\NormalTok{muSimLabel <-}\StringTok{ }\KeywordTok{paste}\NormalTok{(}\StringTok{"sample mean ="}\NormalTok{, mu_sim)}
\KeywordTok{text}\NormalTok{(mu_sim, }\DecValTok{220}\NormalTok{, }\DataTypeTok{labels =}\NormalTok{ muSimLabel, }\DataTypeTok{pos =}\DecValTok{2}\NormalTok{,}
     \DataTypeTok{col =} \StringTok{"red"}\NormalTok{)}
\KeywordTok{abline}\NormalTok{(}\DataTypeTok{v =}\NormalTok{ mu, }\DataTypeTok{col =} \StringTok{"blue"}\NormalTok{)}
\KeywordTok{text}\NormalTok{(mu, }\DecValTok{230}\NormalTok{, }\DataTypeTok{labels =}\NormalTok{ muLabel, }\DataTypeTok{pos =} \DecValTok{2}\NormalTok{, }\DataTypeTok{col =} \StringTok{"blue"}\NormalTok{)}
\end{Highlighting}
\end{Shaded}

\includegraphics{StatInfCP1_files/figure-latex/plotHist-1.pdf}

\begin{Shaded}
\begin{Highlighting}[]
\NormalTok{g <-}\StringTok{ }\KeywordTok{ggplot}\NormalTok{(}\KeywordTok{as.data.frame}\NormalTok{(sm1Kn40), }\KeywordTok{aes}\NormalTok{(sm1Kn40)) }\OperatorTok{+}
\StringTok{        }\KeywordTok{xlab}\NormalTok{(}\StringTok{"Means"}\NormalTok{) }\OperatorTok{+}
\StringTok{        }\KeywordTok{ylab}\NormalTok{(}\StringTok{"Density"}\NormalTok{)}
\NormalTok{g <-}\StringTok{ }\NormalTok{g }\OperatorTok{+}\StringTok{ }\KeywordTok{stat_function}\NormalTok{(}\DataTypeTok{fun =}\NormalTok{ dnorm,}
                       \DataTypeTok{args =} \KeywordTok{list}\NormalTok{(}\DataTypeTok{mean =}\NormalTok{ mu_sim, }
                                   \DataTypeTok{sd =} \KeywordTok{sqrt}\NormalTok{(VarSim)),}
                       \KeywordTok{aes}\NormalTok{(}\DataTypeTok{color =} \StringTok{"Sample Variance"}\NormalTok{), }
                       \DataTypeTok{size =} \DecValTok{1}\NormalTok{) }\OperatorTok{+}
\StringTok{        }\KeywordTok{stat_function}\NormalTok{(}\DataTypeTok{fun =}\NormalTok{ dnorm,}
                      \DataTypeTok{args =} \KeywordTok{list}\NormalTok{(}\DataTypeTok{mean =}\NormalTok{ mu, }
                                  \DataTypeTok{sd =}\NormalTok{ sigma}\OperatorTok{^}\DecValTok{2}\OperatorTok{/}\DecValTok{40}\NormalTok{),}
                      \KeywordTok{aes}\NormalTok{(}\DataTypeTok{color =} \StringTok{"Theoretical Variance"}\NormalTok{), }
                       \DataTypeTok{size =} \DecValTok{1}\NormalTok{)}
\NormalTok{g <-}\StringTok{ }\NormalTok{g }\OperatorTok{+}\StringTok{ }\KeywordTok{geom_vline}\NormalTok{(}\KeywordTok{aes}\NormalTok{(}\DataTypeTok{xintercept =}\NormalTok{ mu_sim, }\DataTypeTok{color =} 
                                \StringTok{"Mean of Sample Means"}\NormalTok{)) }\OperatorTok{+}
\StringTok{        }\KeywordTok{geom_vline}\NormalTok{(}\KeywordTok{aes}\NormalTok{(}\DataTypeTok{xintercept =}\NormalTok{ mu, }\DataTypeTok{color =} 
                               \StringTok{"Mu (Theoretical Mean)"}\NormalTok{))}
\NormalTok{g}
\end{Highlighting}
\end{Shaded}

\includegraphics{StatInfCP1_files/figure-latex/plotVariances-1.pdf}

\hypertarget{conclusions}{%
\subsection{Conclusions}\label{conclusions}}

It was expected, that the mean of sample means would approach the
theoretical mean mu. This is due to the Central Limit Theorem. We also
see the sample variance having similar pattern as the theoretical
variance divided by the number of exponentials. Based on the histogram
of the distribution of the means, we observe that their distribution
looks distinctively Gaussian


\end{document}
